% Please use the skeleton file you have received in the 
% invitation-to-submit email, where your data are already
% filled in. Otherwise please make sure you insert your 
% data according to the instructions in PoSauthmanual.pdf
\documentclass{PoS}

\title{Improved Lattice Renormalization Group Techniques}

\ShortTitle{Improved Lattice RG}

\author{\speaker{First Author}\thanks{A footnote may follow.}\\
        Author affiliation\\
        E-mail: \email{author@email}}

%\author{Another Author\\
%        Affiliation\\
%        E-mail: \email{...}}

\abstract{..........................\
          ...........................}

\FullConference{31st International Symposium on Lattice Field Theory - LATTICE 2013\\
		July 29 - August 3, 2013\\
		Mainz, Germany}


\begin{document}

\section{Introduction}

\section{Monte Carlo Renormalization Group}
Monte Carlo Renormalization Group (MCRG) is a way to study the renormalization properties of a theory.
In MCRG you perform RG transformations to integrate out high momentum modes on the lattice.
Doing so changes the lattice spacing from $a \rightarrow 2a$.
Each iterative RG block transformation flows the system in the infinite dimensinonal coupling space toward the renormalized trajectory in irrelevant directoins and along the renormalized trajectory in relevant directions.
The flow will also go away from ultraviolet fixed points and towards infrared fixed points.
By matching the lattice action $S(\beta_1,n_b) \equiv S(\beta_2,n_{b-1})$, on the renormalized trajectory it is possible to construct the bare step scaling function.
The one hang up with this method is that to find the step scaling function you must be on the renormalized trajectory.
Because our lattices are finite, we have a finite number of blocking steps therefore we must optimize such that we reach the renormalized trajectory in as few steps as possible.
The traditional way of optimizing MCRG is to tune one of the smearing parameters in the block transformation.
This results in a different RG transformation being perfomred at each bare coupling, meaning that each bare coupling is probing a slightly different renormalized trajectory.
In most practical uses of MCRG this has not been a prolbem, however in the study of theories in the conformal window this can blur the location of the infrared fixed point.


\section{Wilson Flow MCRG}
Some paragraph talking about the Wilson Flow.

The properties of the Wilson Flow make it amenable as an optimization step before performing an RG block transformation.
The Wilson flow will continuously move the system in the infitite dimensional coupling space on a surface of constant lattice spacing.
By optimizing a small amount of wilson flow we can come very close to the renormalized trajectory and then preform a fixed blocking step.
Because we are no longer changing the renormalized trajectory in the optimization step we probe a unique renormalized trajectory, allowing us to construct a unique step scaling function.

\section{Procedure}
Our simulation uses a nHPY smeared staggered action with 


\section{Results for 12 Flavors}

\begin{thebibliography}{99}
\bibitem{...} 
....

\end{thebibliography}

\end{document}


