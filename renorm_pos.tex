% Please use the skeleton file you have received in the 
% invitation-to-submit email, where your data are already
% filled in. Otherwise please make sure you insert your 
% data according to the instructions in PoSauthmanual.pdf
\documentclass{PoS}
\usepackage{graphicx}
\usepackage{caption}
\usepackage{subcaption}

\title{Improved Lattice Renormalization Group Techniques}

\ShortTitle{Improved Lattice RG}

\author{\speaker{Gregory Petropoulos}\thanks{A footnote may follow.}\\
        University of Colorado Boulder\\
        E-mail: \email{gregory.petropoulos@colorado.edu}}

\author{Anqi Cheng\\
        University of Colorado Boulder}
\author{Anna Hasenfratz\\
        University of Colorado Boulder}
\author{Another Author\\
        University of Colorado Boulder}



\abstract{
We compute the step scaling function for SU(3) lattice gauge theories with many fundamental fermions using a fully non-perturbative Wilson flow optimized Monte Carlo Renormalization Group technique.
By using Wilson flow to approach the renormalized trajectory of a specific RG transformation we are able to determine the unique step scaling function of a fixed renormalization group scheme.
We apply our Wilson Flow MCRG technique to SU(3) gauge theory with 12 flavors in the chiral limit and find an infrared fixed point. By choosing different renormalization group schemes we are able to move the fixed point but the existence of the fixed point is universal.}

\FullConference{31st International Symposium on Lattice Field Theory - LATTICE 2013\\
		July 29 - August 3, 2013\\
		Mainz, Germany}


\begin{document}

\section{Introduction}
We continue our lattice invstigations of strongly-coupled gauge fermion systems beyond QCD.
This proceedings covers recent improvments that we have made in the method we use to find the bare steps scaling function using Monte Carlo Renormaliztion Group (MCRG) techniques.
We introduce a Wilson Flow optimized MCRG, or WMCRG as an improved tool to find the bare step scaling function.

Section \ref{sec:mcrg} of this proceedings will be a brief overview of MCRG methods.  
Section \ref{sec:wmcrg} discusses optimizing MCRG with the Wilson Flow.  
Section \ref{sec:results} summarizes our study of the step scaling function for SU(3) with 12 flavors of chiral fermions in the fundamental representation using WMCRG.

\section{Monte Carlo Renormalization Group}
\label{sec:mcrg}
The Monte Carlo Renormalization Group is a way to study the renormalization properties of a quantum field theory.
In MCRG you perform RG transformations to integrate out high momentum modes.
On the lattice this is acomplished by performing a block tranformation that changes the lattice spacing from $a \rightarrow 2a$.
Each iterative RG block transformation flows the system in the infinite dimensinonal coupling space.
Flow moves toward the renormalized trajectory in irrelevant directoins and along the renormalized trajectory in relevant directions.
The flow along the renormalized trajectory flows from ultraviolet fixed points and towards infrared fixed points.

By matching the lattice action $S(\beta_1,n_b) \equiv S(\beta_2,n_{b-1})$, on the renormalized trajectory it is possible to construct the bare step scaling function.When the lattice actions are identical, all the observalbes will be identical.
We use the plaquette, the six link loops and the square eight link loop to perform the matching; using small gauge observables we are able to block down to small lattice sizes.
We perform the matching observable by observable, in each case we fit the observable as a function of beta to a function to smooth out statistical fluctuations and so that we can interpolate intermediate values.
We find that a cubic fit is sufficient.

The one difficulty with this method is that to find the step scaling function you must be on the renormalized trajectory.
Because our lattices are finite, we have a finite number of blocking steps.
Therefore we must optimize such that we reach the renormalized trajectory in as few steps as possible.
The traditional way of optimizing MCRG is to tune one of the smearing parameters in the block transformation.
This results in a different RG transformation being perfomred at each bare coupling, meaning that each bare coupling is probing a slightly different renormalized trajectory.
In most practical uses of MCRG this has not been a prolbem, however in the study of theories in the conformal window this can blur the location of the infrared fixed point.



\section{Wilson Flow MCRG}
\label{sec:wmcrg}
\begin{figure}[th]
  \centering
  \includegraphics[height=3in]{fig/wilson_flow_opt.png}
  \caption{insert caption here}
  \label{fig:wflow_opt}
\end{figure}

Some paragraph talking about the Wilson Flow.
Some paragraph talking about the Wilson Flow.
Some paragraph talking about the Wilson Flow.
Some paragraph talking about the Wilson Flow.
Some paragraph talking about the Wilson Flow.

The properties of the Wilson Flow make it amenable as an optimization step before performing an RG block transformation.
The Wilson flow will continuously move the system in the infitite dimensional coupling space on a surface of constant lattice spacing.
By optimizing a small amount of wilson flow we can come very close to the renormalized trajectory and then preform a fixed blocking step.
Because we are no longer changing the renormalized trajectory in the optimization step we probe a unique renormalized trajectory, allowing us to construct a unique step scaling function.



\section{Results for 12 Flavors}
\label{sec:results}
\begin{figure}[h]
  \centering
  \begin{subfigure}[b]{0.48\textwidth}
    \includegraphics[width=\textwidth]{fig/12flav_6-12-24_3lm.pdf}
    \caption{$6^4-12^4-24^4$}
    \label{fig:6-12-24_33}
  \end{subfigure}
  \begin{subfigure}[b]{0.48\textwidth}
    \includegraphics[width=\textwidth]{fig/12flav_8-16-32_44_as_3lm.pdf}
    \caption{$8^4-16^4-32^4$}
    \label{fig:8-16-32_44}
  \end{subfigure}
  \caption{Plot \ref{fig:6-12-24_33} on the left shows the bare step scaling function predicted by three lattice matching with $6^4$, $12^4$, and $24^4$ lattices blocked down to $3^4$.  Plot \ref{fig:8-16-32_44} on the right shows the bare step scaling function predicted by three lattice matching with $8^4$, $16^4$, and $32^4$ lattices blocked down to $4^4$.  Scheme 1 is showen in red, scheme 2 is shown in black and scheme 3 is shown in blue.  The error bars come from the standard deviation of the step scaling predicted by different observables.}
\label{fig:multischeme}
\end{figure}

Our simulation uses a nHPY smeared staggered action with a negative adjoint plaquette term.
The nHYP smearing parameters used in our fermion action are (0.5, 0.5, 0.4).
The adjoint term in our action is included to let us avoid a well-known spurious ultraviolet fixed point caused by lattice artifacts.
The adjoint bare coupling is related to the fundamental bare coupling by $\beta_A=-0.25\beta_F$, and implies that $\beta_F=12/g^2$ at the perturbative level.
To reach the chiral limit our configurations were generated with anti periodic bouldary conditions and zero mass.
All of our analysis was carried out at couplings weaker than the s4 broken phase \cite{}.

We performed three lattice matching with volumes $6^4-12^4-24^4$, $8^4-16^4-32^4$, and $12^4-24^4-48^4$.
We investigated three different scheems by changing the nHYP smearig parameters in our block transformations.
Scheme 1 used nHYP smearing parameters (0.6, 0.2, 0.2), scheme 2 used nHYP smearing parameters (0.6, 0.3, 0.2), and scheme 3 used nHYP smearing parameters (0.65, 0.3, 0.2).

Figure \ref{fig:multischeme} summarizes our results for 12 flavors.
All of the step scaling functions show a clear sign of an infrared fixed point.
We also observe that schemes with more strongly smeared RG block transformations move the fixed point into stronger coupling.
Again we stress that the location of the fixed point is scheme dependant but its existance is not.

Althogh in principle we can block our  $8^4$, $16^4$, and $32^4$ data set one more time to achieve a final blocked lattice size of $2^4$ we are currently unable to reliably predict the step scaling function due to a lack of statistics.
Figure \ref{fig:no22} shows that the blocked plaquette for matching between $32^4$ and $16^4$ volumes blocked down to $2^4$ has large fluctuations.
Although the fits help tame these fluctuations they are large enough to wash out the small values of the step scaling function that we expect from figure \ref{fig:multischeme}.

\begin{figure}[th]
  \centering
  \includegraphics[height=3in]{fig/12flav_8-16-32_3lm_22.pdf}
  \caption{Despite differences in the exact values of the step scaling function predicted by different data sets it is clear that the step scaling starts out negative in strong coupling and ends positively in weak coupling.  The IRFP exists where the step scaling function crosses the x axis.}
  \label{fig:scheme7}
\end{figure}

\section{Conclusion}
\label{sec:conclusion}
Lorem ipsum dolor sit amet, consectetur adipisicing elit, sed do eiusmod tempor incididunt ut labore et dolore magna aliqua.
Ut enim ad minim veniam, quis nostrud exercitation ullamco laboris nisi ut aliquip ex ea commodo consequat.
Duis aute irure dolor in reprehenderit in voluptate velit esse cillum dolore eu fugiat nulla pariatur.
Excepteur sint occaecat cupidatat non proident, sunt in culpa qui officia deserunt mollit anim id est laborum.

Lorem ipsum dolor sit amet, consectetur adipisicing elit, sed do eiusmod tempor incididunt ut labore et dolore magna aliqua.
Ut enim ad minim veniam, quis nostrud exercitation ullamco laboris nisi ut aliquip ex ea commodo consequat.
Duis aute irure dolor in reprehenderit in voluptate velit esse cillum dolore eu fugiat nulla pariatur.
Excepteur sint occaecat cupidatat non proident, sunt in culpa qui officia deserunt mollit anim id est laborum.

\section{Acknowledgments}
\label{sec:acknowledgments}
This research was partially supported by the U.S. Department of Energy (DOE) through Grant No. DE-FG02-04ER41290 (A. C., A. H. and D. S.) and by the DOE Office of Science Graduate Fellowship Program under Contract No. DE-AC05-06OR23100 (G. P.). 
Our code is based in part on the MILC Collaboration’s public lattice gauge theory software.
Numerical calculations were carried out on the HEP-TH and Janus clusters at the University of Colorado; at Fermilab under the auspices of USQCD supported by the DOE; and at the San Diego Computing Center through XSEDE supported by National Science Foundation Grant No. OCI-1053575.

\begin{thebibliography}{99}
\bibitem{...} 
....

\end{thebibliography}

\end{document}
