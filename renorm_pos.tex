% ------------------------------------------------------------------
\documentclass{PoS}
\usepackage{amsmath}
\usepackage{cancel}   % For S4b

% Laziness shortcuts
\newcommand{\be}{\ensuremath{\beta} }
\newcommand{\Sb}{\ensuremath{\cancel{S^4}} }
\newcommand{\MSbar}{\ensuremath{\overline{\textrm{MS}} } }
\newcommand{\fig}[1]{Fig.~\ref{#1}}
\newcommand{\refcite}[1]{Ref.~\cite{#1}}
\newcommand{\secref}[1]{Section~\ref{#1}}
% ------------------------------------------------------------------



% ------------------------------------------------------------------
\title{Improved Lattice Renormalization Group Techniques}
\ShortTitle{Improved Lattice Renormalization Group Techniques}

\author{\speaker{Gregory Petropoulos}, Anqi Cheng, Anna Hasenfratz, David Schaich\footnote{Present address: Department of Physics, Syracuse University, Syracuse, NY 13244} \\
        Department of Physics, University of Colorado, Boulder, CO 80309 \\
        E-mail: \email{gregory.petropoulos@colorado.edu}}

\abstract{ % Draft complete
  We compute the bare step-scaling function $s_b$ for SU(3) lattice gauge theory with $N_f = 12$ massless fundamental fermions, using a fully non-perturbative Wilson-flow-optimized Monte Carlo Renormalization Group technique.
  By applying the Wilson flow to approach the renormalized trajectory of a fixed RG blocking transformation, we are able to determine an $s_b$ corresponding to a unique discrete \be function.
  We carry out this study using new ensembles of 12-flavor gauge configurations generated with exactly massless fermions, finding an $s_b = 0$ infrared fixed point (IRFP) for all investigated lattice volumes and number of blocking steps.
  We also compare different renormalization schemes, each of which indicates an IRFP at a different value of the bare coupling, as expected for an IR-conformal theory.
}

\FullConference{31st International Symposium on Lattice Field Theory - LATTICE 2013 \\
                July 29 -- August 3, 2013 \\
                Mainz, Germany}
% ------------------------------------------------------------------



% ------------------------------------------------------------------
\begin{document}
\section{Introduction} % Draft complete
For the past several years many members of the lattice community have been involved in studying strongly-coupled gauge--fermion systems beyond QCD.
Some of these models may be candidates for new physics beyond the standard model, while others are simply interesting non-perturbative quantum field theories.
Because the dynamics of these systems differ from the familiar case of lattice QCD, it is important to study them with several complementary techniques.
Not only does this allow consistency checks, it can also provide information about the most efficient and reliable methods to investigate non-QCD (e.g., near-conformal) lattice theories.

Monte Carlo Renormalization Group (MCRG) two-lattice matching is one of several analysis tools that we are using to investigate SU(3) gauge theories with many massless fermion flavors.
This technique predicts the step-scaling function $s_b$ in the bare parameter space.
In a previous work~\cite{Petropoulos:2012mg} we proposed an improved MCRG method that exploits the Wilson flow to obtain an $s_b$ corresponding to a unique discrete \be function.
We briefly review our Wilson-flow-optimized MCRG (WMCRG) procedure in Sections~\ref{sec:mcrg}--\ref{sec:wmcrg}.
In \secref{sec:results} we report WMCRG results for SU(3) gauge theory with $N_f = 12$ flavors of massless fermions in the fundamental representation.

This 12-flavor model has been studied by many groups, including Refs.~\cite{Appelquist:2009ty, Deuzeman:2009mh, Fodor:2011tu, Appelquist:2011dp, Hasenfratz:2011xn, DeGrand:2011cu, Cheng:2011ic, Jin:2012dw, Lin:2012iw, Aoki:2012eq, Fodor:2012uw, Fodor:2012et, Itou:2012qn, Cheng:2013eu, Aoki:2013pca, Hasenfratz:2013uha, Hasenfratz:2013eka, Cheng:2013bca}.
Using new ensembles of 12-flavor gauge configurations generated with exactly massless fermions, our improved WMCRG technique predicts a conformal infrared fixed point (IRFP), $s_b = 0$.
This result is stable for all investigated lattice volumes, number of blocking steps and renormalization schemes, and consistent with our other investigations of finite-temperature phase transitions~\cite{Schaich:2012fr, Hasenfratz:2013uha}, the Dirac eigenmode number~\cite{Cheng:2013eu, Cheng:2013bca}, and finite-size scaling~\cite{Hasenfratz:2013eka}.
% ------------------------------------------------------------------



% ------------------------------------------------------------------
\section{\label{sec:mcrg}Monte Carlo Renormalization Group} % Draft complete
MCRG techniques probe lattice field theories by applying RG blocking transformations to integrate out high-momentum (short-distance) modes.
Each blocking step doubles the lattice spacing $a \to 2a$, moving the system in the infinite-dimensional space of lattice-action couplings.
In an IR-conformal system on the $m = 0$ critical surface, a renormalized trajectory runs from the perturbative gaussian FP (where the gauge coupling \be is a relevant operator) to the IRFP (where \be is irrelevant).\footnote{Because the locations of these fixed points in the action-space depend on the renormalization scheme, each scheme corresponds to a different renormalized trajectory.}
The RG flow produced by the blocking steps moves the system towards and along the renormalized trajectory.
At stronger couplings, where we would na\"ively expect backward flow, there might be no ultraviolet FP to drive the RG flow along a renormalized trajectory.
Except in the immediate vicinity of the IRFP, every method that attempts to determine the strong-coupling flow of the gauge coupling (including MCRG two-lattice matching) might then become meaningless.

By matching the lattice actions $S(\be_1, n_b)$ and $S(\be_2, n_{b - 1})$ for systems with bare couplings $\{\be_1, \be_2\}$ after $\{n_b, n_{b - 1}\}$ blocking steps, it is possible to determine the bare step-scaling function $s_b(\be_1) \equiv \lim_{n_b \to \infty} \be_1 - \be_2$.
See \refcite{Petropoulos:2012mg} for more details.
When the lattice actions are identical, all the observables will be identical, so we use the plaquette, the three six-link loops and a planar eight-link loop to perform this matching.
Using short-distance gauge observables allows us to carry out more blocking steps, down to small $2^4$ or $3^4$ lattices.
We perform the matching independently for each observable, fitting the data as a cubic function of \be to smoothly interpolate between investigated values of the gauge coupling.

Our finite lattices only allow a few blocking steps, so we must optimize the two-lattice matching to reach the renormalized trajectory in as few steps as possible.
In practice, we optimize by tuning some parameter so that consecutive RG blocking steps yield the same $\be_1 - \be_2$, which we identify as $s_b(\be_1)$.
Traditional optimization tunes the RG blocking transformation.
This results in a different renormalization scheme being probed for each bare coupling $\be_1$, meaning that the $s_b$ we obtain is a composite of many different discrete \be functions.
The Wilson flow provides a parameter that we can tune without changing the scheme.
% ------------------------------------------------------------------



% ------------------------------------------------------------------
\section{\label{sec:wmcrg}Wilson-flow-optimized MCRG} % Draft complete
\begin{figure}[th]
  \centering
  \includegraphics[height=3 in]{fig/wilson_flow_opt.png}
  \caption{The Wilson flow (blue) moves systems on a surface of constant lattice scale $a$ (normal to the orange renormalized trajectory) in the infinite-dimensional coupling space.  Wilson-flow-optimized MCRG tunes the flow time to bring the system close to the renormalized trajectory (yellow star), so that MCRG blocking (green) quickly reaches the renormalized trajectory.}
  \label{fig:wflow_opt}
\end{figure}

The Wilson flow is a continuous smearing transformation~\cite{Narayanan:2006rf} that removes UV fluctuations without changing the lattice scale, as shown in \fig{fig:wflow_opt}.
In perturbation theory it is related to the \MSbar running coupling~\cite{Luscher:2010iy}, and can be used to compute a renormalized step-scaling function~\cite{Fodor:2012td, Fodor:2012qh}.
While we are exploring this approach, it is based on perturbative relations that are only fully reliable at weak coupling, so in this proceedings focus on fully non-perturbative MCRG methods.

We use the Wilson flow to optimize MCRG two-lattice matching with a fixed RG blocking transformation (renormalization scheme).
The Wilson flow continuously moves the system on a surface of constant lattice scale in the infinite-dimensional space of lattice-action couplings.
We tune the flow time to bring the system as close as possible to the renormalized trajectory.
After running the optimal amount of Wilson flow on the unblocked lattices, we then carry out the MCRG two-lattice matching.
Because the renormalization scheme is fixed, we obtain a bare step-scaling function that corresponds to a unique discrete \be function.
% ------------------------------------------------------------------



% ------------------------------------------------------------------
\section{\label{sec:results}Results for 12 Flavors} % Draft complete
\begin{figure}[ht]
  \centering
  \includegraphics[height=3 in]{fig/12flav_6-12-24_3lm}
  \caption{The bare step-scaling function $s_b$ predicted by three-lattice matching with $6^4$, $12^4$ and $24^4$ lattices blocked down to $3^4$, comparing three different renormalization schemes.  The error bars come from the standard deviation of predictions using the different observables discussed in \protect\secref{sec:mcrg}.}
  \label{fig:24to3}
\end{figure}

\begin{figure}[ht]
  \centering
  \includegraphics[height=3 in]{fig/12flav_8-16-32_44_as_3lm}
  \caption{As in \protect\fig{fig:24to3}, the bare step-scaling function $s_b$ for three different renormalization schemes from three-lattice matching, now using $8^4$, $16^4$ and $32^4$ lattices blocked down to $4^4$.}
  \label{fig:32to4}
\end{figure}

We now present WMCRG results for the 12-flavor system, using new ensembles of gauge configurations generated with exactly massless fermions.
Our lattice action uses nHYP-smeared staggered fermions with smearing parameters (0.5, 0.5, 0.4), and a negative adjoint plaquette term in the gauge action.
The adjoint plaquette term moves our systems away from a well-known spurious ultraviolet fixed point caused by lattice artifacts.
The adjoint bare coupling is related to the fundamental bare coupling by $\beta_A = -0.25\beta_F$, which implies that $\beta_F = 12 / g^2$ at the perturbative level.
To run with $m = 0$ we employ anti-periodic boundary conditions in all four directions.
All of our analyses are carried out at couplings weak enough to avoid the unusual strong-coupling ``$\Sb$'' phase discussed by Refs.~\cite{Cheng:2011ic, Hasenfratz:2013uha}.

We perform three-lattice matching~\cite{Hasenfratz:2011xn} with volumes $6^4$--$12^4$--$24^4$ and $8^4$--$16^4$--$32^4$.
Three-lattice matching carries out two sequential two-lattice matching steps, to minimize finite-volume effects.
Both two-lattice matching steps are carried out on the same final volume $V_f$.
For example, $6^4$--$12^4$--$24^4$ three-lattice matching compares $6^4$ and $12^4$ lattices blocked down to $V_f = 3^4$, then matches $12^4$ and $24^4$ blocked to $3^4$.
The number of blocking steps on the largest volume is $n_b$, in this case $n_b = 3$.
Using the $8^4$--$16^4$--$32^4$ data we determine the bare step-scaling function for $n_b = 3$ and $V_f = 4^4$ as well as $n_b = 4$ and $V_f = 2^4$.
We investigate three renormalization schemes by changing the HYP smearing parameters in our blocking transformation: scheme~1 uses smearing parameters (0.6, 0.2, 0.2), scheme~2 uses (0.6, 0.3, 0.2) and scheme~3 uses (0.65, 0.3, 0.2).

Figs.~\ref{fig:24to3}, \ref{fig:32to4} and \ref{fig:scheme1} present representative results for 12 flavors.
All of the bare step-scaling functions clearly show $s_b = 0$, signalling an infrared fixed point, for every $n_b$, $V_f$ and renormalization scheme.
Appropriately for an IR-conformal system, the location of the fixed point is scheme dependent.
We observe that the fixed point moves to stronger coupling as the HYP smearing parameters in the RG blocking transformation increase.

When we block our $8^4$, $16^4$ and $32^4$ lattices down to a final volume $V_f = 2^4$ (corresponding to $n_b = 4$), the observables become very noisy, making matching more difficult.
The problem grows worse as the HYP smearing parameters increase, and our current statistics do not allow reliable three-lattice matching in schemes~2 and 3.
To resolve this issue, we are accumulating more statistics in existing $32^4$ runs, and also generating additional $32^4$ ensembles at more values of the gauge coupling $\be$.
These additional data will also improve our results for scheme~1, which we show in \fig{fig:scheme1}.
Different volumes and $n_b$ do not produce identical results in scheme~1, suggesting that the corresponding systematic effects are still non-negligible.
We can estimate finite-volume effects by comparing $n_b = 3$ with $V_f = 3^4$ and $V_f = 4^4$.
Systematic effects due to $n_b$ can be estimated from $n_b = 4$ and $V_f = 2^4$, but this is difficult due to the noise in the $2^4$ data.
Even treating the spread in the results shown in \fig{fig:scheme1} as a systematic uncertainty, we still obtain a clear zero in the bare step-scaling function, indicating an IR fixed point.

\begin{figure}[th]
  \centering
  \includegraphics[height=3 in]{fig/12flav_8-16-32_3lm_22}
  \caption{The bare step-scaling function $s_b$ for scheme~1, comparing three-lattice matching using different volumes: $6^4$, $12^4$ and $24^4$ lattices blocked down to $3^4$ (black $\times$s) as well as $8^4$, $16^4$ and $32^4$ lattices blocked down to $4^4$ (blue bursts) and $2^4$ (red crosses).}
  \label{fig:scheme1}
\end{figure}
% ------------------------------------------------------------------



% ------------------------------------------------------------------
\section{Conclusion} % Draft complete
In this proceedings we have shown how the Wilson-flow-optimized MCRG two-lattice matching procedure proposed in \refcite{Petropoulos:2012mg} improves upon traditional lattice renormalization group techniques.
By optimizing the flow time for a fixed RG blocking transformation, WMCRG predicts an $s_b$ that corresponds to a unique discrete \be function.
Applying WMCRG to new 12-flavor ensembles generated with exactly massless fermions, we observe an infrared fixed point in the bare step-scaling function $s_b$.
The fixed point is present for all investigated lattice volumes, number of blocking steps and renormalization schemes, even after accounting for systematic effects indicated by \fig{fig:scheme1}.
This result reinforces the IR-conformal interpretation of $N_f = 12$ suggested by our complementary studies of phase transitions~\cite{Schaich:2012fr, Hasenfratz:2013uha}, the Dirac eigenmode number~\cite{Cheng:2013eu, Cheng:2013bca}, and finite-size scaling~\cite{Hasenfratz:2013eka}.
% Something about accumulating more data?  Or would that be less of a high note to go out on?
% ------------------------------------------------------------------



% ------------------------------------------------------------------
\section*{Acknowledgments} % Draft complete
This research was partially supported by the U.S.~Department of Energy (DOE) through Grant No.~DE-SC0010005 (A.~C., A.~H.\ and D.~S.) and by the DOE Office of Science Graduate Fellowship Program under Contract No.~DE-AC05-06OR23100 (G.~P.).
Our code is based in part on the MILC Collaboration's public lattice gauge theory software.\footnote{\href{http://www.physics.utah.edu/~detar/milc/}{http://www.physics.utah.edu/$\sim$detar/milc/}}
Numerical calculations were carried out on the HEP-TH and Janus clusters at the University of Colorado, the latter supported by the U.S.~National Science Foundation (NSF) through Grant No.~CNS-0821794; at Fermilab under the auspices of USQCD supported by the DOE; and at the San Diego Computing Center and Texas Advanced Computing Center through XSEDE supported by NSF Grant No.~OCI-1053575.
% ------------------------------------------------------------------



% ------------------------------------------------------------------
\bibliographystyle{utphys}
\bibliography{renorm_pos}
\end{document}
% ------------------------------------------------------------------
